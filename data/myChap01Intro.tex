\chapter{引言}
\label{cha:intro}

\section{研究背景}
%background info here
对于目标的实时位置的获取是无线通信技术的应用的重要一部分\cite{indoorPos},其在车辆导航与编队、军事演习等方面有着广泛的应用前景。


%tech info here
通常一个无线定位系统可以依赖于GPS卫星定位,但在室内定位的场合,由于微波被建筑物散射等原因,定位效果并不理想,在这种情况下需要根据应用的场景开发
地面无线定位系统。通常一个地面无线定位系统会事先部署一些位置已知的锚点(基站)采用某个特殊的频段的电磁波与位置未知的目标节点进行通信,在锚点处可以通过测量无线信号到达的时间或信号强度等信息估计出某个基站与场景中目标节点的距离,如果是对目标的追踪也会利用上前一个时刻得到的定位结果,利用统计学的方法对这些数据进行实时的处理,可以估计出目标节点的位置或运动的轨迹。


%non-cooperative case to cooperative case
传统的定位方法是只利用锚点和目标节点彼此之间的测距信息对目标节点进行定位,在有多个目标节点的定位场景下,随着技术的成熟近年来发展出了利用目标节点之间相互通信得到的距离信息的协作定位技术,协作定位技术不仅利用了目标节点和已知位置的锚点之间的信息,还利用了目标节点彼此之间的定位信息,从而提高了定位精度。


%algorithm property analysis
协作定位技术已经有了一定的研究基础,目前已经有大量的文献针对定位算法展开探讨,对定位算法的性能研究一般可以通过理论推导和仿真比较等方法,在理论
推导方面,已经得出了在给定的定位场景(定位网络)下存在一个统计平均意义上的定位误差下界\cite{LimitBound},任何基于测量数据对目标节点的位置估计的误差都在这个定位误差下界之上。因此研究和分析这个定位误差下界随着定位网络规模的扩大和采样时间间隔的缩短对于定位算法的设计具有一定的指导意义。

\section{研究问题}
%content
在本人的研究中,我会首先建立定位网络的数学模型并根据建立的模型推导定位误差下界的一般表达式,然后分别针对若干特殊的定位网络推导误差下界的解析表达式,并根据解析表达式辅助以必要的数值计算分析定位性能随网络的时空规模的变化规律。

%focus
\section{文章结构}
本文的研究重点是特殊定位网络误差下界的推导,在第(\ref{cha:model})章中给出了问题的数学模型,主要分非协作定位场景(\ref{section:noncooperative_localization})、空间协作定位场景(\ref{section:cooperative_localization})、时间协作定位场景(\ref{section:temporal_cooperative_localization})三部分,(\ref{section:model_discussion})一节中对模型的合理性进行了进一步讨论。
在第(\ref{cha:content3})章中分别对上一节提出的数学模型进行初步的分析和求解,其中非协作情形(\ref{section:circle_general})针对小节(\ref{section:noncooperative_localization})的模型,空间两个节点协作(\ref{section:two_node_cooperation})
针对(\ref{section:cooperative_localization})的模型。
在第(\ref{cha:content4})章中分别对上一节提出的数学模型进行深入的分析和求解,其中N个节点两两协作(\ref{section:complete_graph_cooperation})、大规模正方形和正六边形网络协作(\ref{section:square_and_hexagon_network})
针对(\ref{section:cooperative_localization})的模型,一个节点时间上的协作(\ref{section:linear_network})针对(\ref{section:temporal_cooperative_localization})的模型。
最后第(\ref{cha:content5})章对全文使用的数学方法和取得的成果进行了总结,