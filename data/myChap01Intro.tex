\chapter{引言}
\label{cha:intro}

\section{研究背景}
%background info here
对于目标的实时位置的获取是无线通信技术的应用的重要一部分\cite{indoorPos},其在仓库管理、车辆导航与编队、军事演习等方面有着广泛的应用前景。


%tech info here
通常一个无线定位系统可以依赖于GPS卫星定位,但在室内定位的场合,由于微波被建筑物散射等原因,定位效果并不理想,在这种情况下需要根据应用的场景开发
地面无线定位系统。通常一个地面无线定位系统会事先部署一些位置已知的锚点(基站)采用某个特殊的频段的电磁波如超宽带(Ultra Wideband)与位置未知的目标节点进行通信,在锚点处可以通过测量无线信号到达的时间或信号强度等信息估计出某个基站与场景中目标节点的距离,
%如果是对目标的追踪的情形可以利用两个时刻间速度的测量结果得到距离信息,
利用统计学的方法对这些数据进行实时的处理,可以估计出目标节点的位置或运动的轨迹。


%non-cooperative case to cooperative case
传统的定位方法是只利用锚点和目标节点彼此之间的测距信息对目标节点进行定位,在一些环境比较苛刻的定位场景,要达到较高的定位精度锚点部署的密度和功耗都要比较大,因而需要的成本也随之提高。在有多个目标节点的定位场景下,随着技术的成熟近年来发展出了利用目标节点之间相互通信得到的距离信息的协作定位技术,协作定位技术不仅利用了目标节点和已知位置的锚点之间的信息,还利用了目标节点彼此之间的定位信息,不仅可以提高定位精度,也降低了开销。
已知协作网络中的测量信息后,网络中各个节点在上一轮的位置信息可以用来校准它的邻居节点的位置以提高定位精度,多轮的协作使网络中的节点可以和超出其通信范围的其他较远的节点间接地协作,并最终达到稳态,这是静态协作网络的典型场景。
在对节点的位置有实时性要求的场合,网络中各个节点还可以利用自身之前时刻位置的估计值来对自身当前时刻进行定位,这方面比较著名的算法有卡尔曼滤波\cite{KF}等方法。

%algorithm property analysis
协作定位技术已经有了一定的研究基础,目前已经有大量的文献针对定位算法展开探讨,对定位算法的性能分析需要对定位误差能够量化,定位误差最常采用的方法是使用估计量的均方误差来量化。如果估计量是无偏的,可以知道估计量的均方误差总是在克拉美罗界之上\cite{si}。克拉美罗界是根据估计量的费舍尔信息量计算出来的均方误差的理论下界,因此费舍尔信息量可以作为衡量定位精度的理论上界。在本文中我们将主要从费舍尔信息量的角度研究定位误差。

从原始信号中可以提取的观测数据的有多种,一般常用的有信号到达时间(time-of-arrival),信号到度角度(angle-of-arrival),以及接收信号强度(received-signal-strength)等。基于这些不同的观测数据建立的模型有一定区别,由此导出的定位误差下界也不相同。文献\cite{LimitBound}针对不同的观测量研究了定位误差下界的规律,在本文的研究中主要针对信号到达时间这一观测,因为由信号到达时间乘以光速即可得到距离的测量量,所以本文在数学模型第\ref{cha:model}章中直接假设距离是观测量。

实际定位系统中无线信号受环境条件的限制,传播过程中会出现非视距(non-light-of-sight)和多径(multi-path)等问题\cite{indoorPos:differentpage},这给定位误差的分析增加了难度,为了反映实际,通常在数学模型中会把由非视距和多径效应带来的额外信道参数作为未知的估计量与节点的位置估计量一并考虑\cite{LimitBound}。另一个性能瓶颈是不同节点之间彼此的数字时钟不一定完全同步(clock synchronization)而引入的测距误差\cite{indoorPos:differentpage2},时钟同步问题一方面可以通过系统设计尽量减轻其带来的影响,另一方面也可以用统计的方法将时钟偏移作为未知参数,在数据处理阶段进行处理来降低时钟不同步带来的定位误差。本文中的模型为能够研究问题的机理而建立的模型比较简单,并不考虑这些信道参数带来的影响。

对定位误差理论下界的研究工作不仅可以为不同的定位算法提供可以参照的最优定位结果,也可以指导定位网络中节点的部署。在研究方法方面可以采用理论推导和仿真比较相结合的方法:在理论推导方面,文献\cite{LimitBound}探讨了多径和非视距传播对定位误差下界的影响,并推导了定位误差下界的上下界;对于定位误差界随链路衰减的特征在文献\cite{siyi}中已经有了一些比较宏观的刻画。本文在文献\cite{LimitBound},\cite{siyi}的研究工作基础上进一步分析这个定位误差下界随着定位网络规模的扩大和采样时间间隔的缩短的规律,探究协作定理网络的信息耦合机理。

%research of others in synthesis
%  在文献\cite{LimitBound2}中为研究定位误差下界提出了等效费舍尔信息矩阵、信息椭圆等概念,同时对两个节点协作的简单场景分析了定位误差和定位%信息强度的关系;本文中会引用\cite{LimitBound2}中相关的结论,同时在必要之处会采用新的数学方法给出更加简洁的结论推导。
\section{研究问题}
%content
%research current state here
在本人的研究中,会首先建立定位网络的数学模型并根据建立的模型推导定位误差下界的一般表达式,然后分别针对若干特殊的定位网络推导误差下界的解析表达式,并根据解析表达式辅助以必要的数值计算分析定位性能随网络的时空规模的变化规律。

%focus
\section{文章结构}
本文的研究重点是特殊定位网络误差下界的推导,在第(\ref{cha:model})章中给出了问题的数学模型,主要分非协作定位场景(\ref{section:noncooperative_localization})、空间协作定位场景(\ref{section:cooperative_localization})、时间协作定位场景(\ref{section:temporal_cooperative_localization})三部分,(\ref{section:model_discussion})一节中对模型的合理性进行了进一步讨论。
在第(\ref{cha:content3})章中分别对上一节提出的数学模型进行初步的分析和求解,其中非协作情形(\ref{section:circle_general})针对小节(\ref{section:noncooperative_localization})的模型,空间两个节点协作(\ref{section:two_node_cooperation})
针对(\ref{section:cooperative_localization})的模型。
在第(\ref{cha:content4})章中分别对上一节提出的数学模型进行深入的分析和求解,其中N个节点两两协作(\ref{section:complete_graph_cooperation})、大规模正方形和正六边形网络协作(\ref{section:square_and_hexagon_network})
针对(\ref{section:cooperative_localization})的模型,一个节点时间上的协作(\ref{section:linear_network})针对(\ref{section:temporal_cooperative_localization})的模型。
最后第(\ref{cha:content5})章对全文使用的数学方法和取得的成果进行了总结。