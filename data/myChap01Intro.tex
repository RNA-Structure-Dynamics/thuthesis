\chapter{引言}
\label{cha:intro}

\section{研究背景}
%background info here
对于目标的实时位置的获取是无线通信技术的应用的重要一部分,其在车辆导航与编队、军事演习等方面有着广泛的应用前景。


%tech info here
通常一个无线定位系统可以依赖于GPS卫星定位,但在室内定位的场合,由于微波被建筑物散射等原因,定位效果并不理想,在这种情况下需要根据应用的场景开发
地面无线定位系统。通常一个地面无线定位系统会事先部署一些位置已知的锚点(基站)采用某个特殊的频段的电磁波与位置未知的目标节点进行通信,在锚点处可以通过测量无线信号到达的时间或信号强度等信息估计出某个基站与场景中目标节点的距离,如果是对目标的追踪也会利用上前一个时刻得到的定位结果,利用统计学的方法对这些数据进行实时的处理,可以估计出目标节点的位置或运动的轨迹。


%non-cooperative case to cooperative case
传统的定位方法是只利用锚点和目标节点彼此之间的测距信息对目标节点进行定位,在有多个目标节点的定位场景下,随着技术的成熟近年来发展出了利用目标节点之间相互通信得到的距离信息的协作定位技术,协作定位技术不仅利用了目标节点和已知位置的锚点之间的信息,还利用了目标节点彼此之间的定位信息,从而提高了定位精度。


%algorithm property analysis
协作定位技术已经有了一定的研究基础,目前已经有大量的文献针对定位算法展开探讨,对定位算法的性能研究一般可以通过理论推导和仿真比较等方法,在理论
推导方面,已经得出了在给定的定位场景(定位网络)下存在一个统计平均意义上的定位误差下界\cite{LimitBound},任何基于测量数据对目标节点的位置估计的误差都在这个定位误差下界之上。因此研究和分析这个定位误差下界随着定位网络规模的扩大和时间上的延长对于定位算法的设计具有一定的指导意义。

\section{研究问题}
%content
在本人的研究中,我会首先建立定位网络的数学模型并根据建立的模型推导定位误差下界,然后分别针对若干特殊的定位网络推导误差下界的解析表达式,并根据解析表达式分析定位性能随网络的时空规模的变化规律。

%focus
\section{文章结构}
本文的研究重点是特殊定位网络误差下界的推导,该推导首先要处理对称正定的矩阵,因此本文会应用已有的结果如求一个矩阵的逆阵的某个子式的公式,关于矩阵逆的恒等式等结论来对要研究的矩阵进行预处理;其次由于很多时候直接推导存在很大的困难,需要先归纳后证明,本文充分利用了数学归纳法完成这一任务;最后推导得出的表达式往往比较复杂,表达式的化简需要特殊的数学方法,本文根据问题的特征分别采用了瑞利商、黎曼积分以及连分式等思路。
