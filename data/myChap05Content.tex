\chapter{结论}\label{cha:content5}

  % Keep the summary *very short*.
  \begin{itemize}
  \item
  数学方法方面
  \begin{itemize}
  \item 本文的推导首先要处理对称正定的矩阵,因此本文会应用已有的结果如求一个矩阵的逆阵的某个子式的公式、关于矩阵逆的恒等式等结论来对要研究的矩阵进行预处理
  \item 其次由于很多时候直接推导存在很大的困难,需要先归纳后证明,本文充分利用了数学归纳法完成这一任务
  \item 最后推导得出的表达式往往比较复杂,表达式的化简需要特殊的数学方法,本文根据问题的特征分别采用了瑞利商、黎曼积分以及连分式等思路
  \end{itemize}
  \item
    已取得的成果
  \begin{itemize}
  \item
    使用复数表示法推导得出非协作定位场景下费舍尔信息矩阵的特征值和特征向量的表达式.
  \item
    推导得出秩一矩阵的克罗内克积对N维对称正定矩阵扰动后行列式的表达式
  \item
    推导得出二维场景下特殊完全图的邻接矩阵所有特征值,其中使用瑞利商给出了最大 特征值的表达式
  \item 推导得出二维场景下特殊度为2的图的邻接矩阵的所有特征值;当网络规模趋向无穷大时,求出了所有特征值的倒数和的平均值的极限
  \item 使用连分式推导得出形如式(\ref{eq:Pab})的对称正定矩阵$\bm{A}$确定的$\bm{A}^{-1}_{1\times2,1\times2}$的两个特征值;分析得出了决定特征值的连分式的序列指数收敛的特性,并做出适当的推广。
  \end{itemize}
  \item \textbf{未来展望}\\
  本文着重于对网络中信息耦合机理与网络中角度这个几何参量的分析,所研究的模型比较简单,可能与实际问题有一定的出入。今后的工作可以结合计算机仿真工具对复杂网络拓扑下的定位误差作深入的探讨。
  \end{itemize}
